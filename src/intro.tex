線形計画問題において、

Chubanovが$m \times n$行列$A$と$\mathbf{x} \in \mathbb{R}^n$に対して、$A \mathbf{x} = 0, \mathbf{x} \geq 0$という問題の許容解かそのような許容解が存在しないかを求めることができる強多項式時間アルゴリズムを提案した\cite{Chubanov}。Chubanovのアルゴリズムでは次々と凸結合をつくっていくBasic ProcedureとBasic Procedureを呼び出し、その返り値に基づいて制約中にでてくる$A$をスケールするMain Procedureを組み合わせて、$\mathbf{x} > \mathbf{0}$となる許容解かそれともそのような許容解が存在しないかを導く。

村松らがそのアルゴリズムを許容解が対称錐内にあって線型写像によって写像した結果が$\mathbf{0}$になるような問題に適用できるように拡張した\cite{SymmetricCone}。対称錐は半正定値な行列の集合でできた錐を含むので、村松らが拡張したアルゴリズムを半正定値な行列の集合でできた錐を制約としてもつ半正定値計画問題に対しても適用することができる。そこで、本論文では$n$次実対称行列$X, Y$に対して、行列の内積を行列のトレース$\mathrm{tr}$を用いて、
\begin{align*}
  X \cdot Y = \mathrm{tr}(X Y)
\end{align*}
と定義し、さらに行列$A$が正定値、すなわち
\begin{align}
  \text{任意の} \mathbf{0} \text{でない} \mathbf{x}^T \text{に対して} \mathbf{x}^T A \mathbf{x} > 0 \label{PD}
\end{align}
を$A \succ 0$と定義し、村松らが対称錐へと拡張したChubanovのアルゴリズムを半正定値な行列の集合がなす錐に制限した
\begin{align} \label{SemidefiniteSystem}
  \begin{array}{ll} \\
    \text{find} & X \\
    \text{s.t.} & A_i \cdot X = 0 \,\, (i = 1, 2, \cdots, m) \\
                & X \succ 0
  \end{array}
\end{align}
という問題における許容解$X \succ 0$が存在するかを導けるように拡張したアルゴリズムの実装・実験を行った。
