線形計画問題において、
\begin{itemize}
  \item 内点法(interior point method)
  \item 単体法(simplex method)
\end{itemize}
という2つのアルゴリズムが最適値を求めるための効率の良いアルゴリズムとして長らく使用されている。

単体法とは線形計画問題を不等式標準形、等式標準形と変形していき、等式標準形の等式制約を「辞書」としてこの辞書を非増加となるように更新していくアルゴリズムである。このアルゴリズムは幾何的な許容領域上の頂点をたどるように動作する。欠点として、辞書が変化しなくなる「非退化」という現象が現れてしまうことがある。これを防ぐために最大添え字規則やBlandのピボット規則などが考え出されている\cite*{Optimization}。

内点法とはカーマーカー法をはじめとする一連のアルゴリズムの総称であり、単体法の許容領域上の頂点をたどるように動作するのとは違い、許容領域の内点をたどりながら最適解に収束するように動作する。

どちらのアルゴリズムも効率が良いため、様々な改良が提案されている\cite*{InteriorPointMethod}。

Chubanovはこれらのアルゴリズムに対して新しいアルゴリズムを提案した\cite*{Chubanov}。その新しいアルゴリズムとは単体法と内点法の2つのように線形計画問題を解いて最適解を求めるのではなく、線形計画問題の許容解を求めるアルゴリズムである。そのChubanovの提案したアルゴリズムは、$m \times n$行列$A$と$\mathbf{x} \in \mathbb{R}^n$に対して、
\begin{align*}
  A \mathbf{x} = \mathbf{0}, \mathbf{x} \geq \mathbf{0}
\end{align*}
という同次な不等式制約を満たすような$\mathbf{x}$を探すために次々と凸結合をつくっていくBasic Procedureと、Basic Procedureを呼び出してその返り値に基づいて制約中にでてくる$A$をスケールするMain Procedureの2つを組み合わせて同次な不等式制約をもつ線形システムを多項式時間で解く。

線形システムの拡張として2次錐の内点となる点を探すものやや本論文で取り扱う半正定値錐の内点を探すもの、対称錐の内点を探すものがあり、このChubanovが提案したアルゴリズムをこれらの線形システムを拡張した問題に拡張する研究がなされている\cite*{SOCP}\cite*{SymmetricCone}。

Louren\c{c}oらはChubanovのアルゴリズムを許容解が対称錐における内点許容解を探し出せるアルゴリズムとして拡張した\cite*{SymmetricCone}。対称錐計における線形システムの拡張は
\begin{align*}
  \begin{array}{ll}
    \text{min}  & \dot<\mathbf{c}, \mathbf{x}> \\
    \text{s.t.} & \dot<\mathbf{a}_i, \mathbf{x}> = b_i \,\, (i = 1, 2, \cdots, m) \\
                & \mathbf{x} \in \mathcal{K}
  \end{array}
\end{align*}
となる。$\mathcal{K}$は対称錐を表しており、$\mathbf{a}_i$は線形写像を、$\dot<\cdot, \cdot>$は内積を表している。この対称錐システムは半正定値システムを部分クラスとして含むため、Louren\c{c}oらが提案した対称錐システムにおけるChubanovのアルゴリズムの拡張を半正定値システムに対しても制限した形で適用することが可能である。

本論文ではそのLouren\c{c}oらが拡張した対称錐システムにおけるChubanovのアルゴリズムの拡張を半正定値錐を制約としてもつ半正定値計システムに対して制限したアルゴリズムを実装し、その効率がどれほどのものであるかを実装し、確認した。
