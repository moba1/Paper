錐$\mathcal{K} \subseteq \mathbb{R}^n$とは任意の$\mathbf{x} \in \mathcal{K}, \alpha > 0$に対して、$\alpha \mathbf{x} \in \mathcal{K}$となるような集合のことをいい、集合$\mathcal{K}$が凸であるとは
\begin{align*}
  \mathbf{x}_1, \mathbf{x}_2 \in \mathcal{K}, \alpha_1, \alpha_2 \in \mathbb{R}
\end{align*}
に対して
\begin{align*}
  \alpha_1 + \alpha_2 = 1, \alpha_1, \alpha_2 \geq 0
\end{align*}
としたとき
\begin{align*}
  \alpha_1 \mathbf{x}_1 + \alpha_2 \mathbf{x}_2 \in \mathcal{K}
\end{align*}
となることをいう。凸な錐は$\mathcal{K}$を凸錐と呼ばれ、その凸錐が閉集合の時、その錐のことを閉凸錐と呼ばれる。

$S_+^n$はこの定義のもと凸錐となる。実際、
\begin{theorem*}
  $S_+^n$は凸錐である。
\end{theorem*}
\begin{proof}
  まず、$S_+^n$が錐であることを示す。$\alpha > 0$と任意のベクトル$\mathbf{x} \in \mathbb{R}^n$および$A \in S_+^n$に対して、$\alpha A$が$S_+^n$に入るかを確認する。そのために$\mathbf{x}^T \alpha A \mathbf{x}$が非負であるかを確かめる。まず、$\alpha$が非負の実数であることから
  \begin{align*}
    \mathbf{x}^T \alpha A \mathbf{x} = \alpha \mathbf{x}^T A \mathbf{x}
  \end{align*}
  となり、$A \in S_+^n$、$\alpha > 0$という仮定から$\mathbf{x}^T A \mathbf{x} \geq 0$より、$\alpha \mathbf{x}^T A \mathbf{x} \geq 0$。従って$\alpha A \in S_+^n$となるので$S_+^n$は錐である。

  次に、$S_+^n$が凸であることを示す。任意の$A, B \in S_+^n$、$\alpha, \beta \geq 0$に対して、$\alpha + \beta = 1$とする。任意のベクトル$\mathbf{x} \in \mathbf{R}^n$に対して、
  \begin{align*}
    \mathbf{x}^T (\alpha A + \beta B) \mathbf{x} = \alpha \mathbf{x}^T A \mathbf{x} + \beta \mathbf{x}^T B \mathbf{x}
  \end{align*}
  。ここで、仮定$A, B \in S_+^n$、つまり$\mathbf{x}^T A \mathbf{x} \geq 0, \mathbf{x}^T B \mathbf{x} \geq 0$および$\alpha, \beta \geq 0$から$\alpha \mathbf{x}^T A \mathbf{x} \geq 0$かつ$\beta \mathbf{x}^T B \mathbf{x} \geq 0$より、$\alpha \mathbf{x}^T A \mathbf{x} + \beta \mathbf{x}^T B \mathbf{x} \geq 0$。したがって、$\alpha \mathbf{x}^T A \mathbf{x} + \beta \mathbf{x}^T B \mathbf{x} \in S_+^n$。ゆえに$S_+^n$は凸である。

  以上から集合$S_+^n$は凸かつ錐となるので、$S_+^n$は凸錐である。
\end{proof}
という定理から凸錐である事実がわかる。
\begin{figure}
  \includegraphics{PSDcone.png}
  \caption[]{$\displaystyle{S_+^2 = \Set{
                                      \left(
                                        \begin{array}{cc}
                                          x & z \\
                                          z & y
                                        \end{array}
                                      \right)
                                      | x \geq 0, z \geq 0, xz \geq y^2
                                    }}$}
\end{figure}

