アルゴリズムを実装し、各データセットにおけるアルゴリズムの実行時間をそれぞれ示す。実験環境は
\begin{description}
  \item[OS] Mac OS X El Caption (10.11.6)
  \item[CPU] Intel(R) Core(TM) i7-3770 CPU @ 3.40GHz
  \item[Memory] 16GB 1600MHz DDR3
\end{description}
でおこない、
\begin{algorithm}
  \caption{$\epsilon > 0$の与え方}
  \begin{algorithmic}
    \For {$i = 1$ to $10$}
      \For {$j = 9$ to $1$}
        \State $\epsilon = j \times 10^{-i}$
        \For {$k = 1$ to $100$}
          \State Main Procedureを$\epsilon$と実験で用いる$A_1, A_2, \cdots, A_m$を引数に呼び出し、実行時間を計測
        \EndFor
      \EndFor
    \EndFor
  \end{algorithmic}
\end{algorithm}
として、Forループ1反復ごとの$\epsilon$とともにMain Procedureを100回呼び出してMain Procedureの実行時間から平均実行時間を算出した。実装はMATLAB 2016bを用いて行った。

\subsection{実験1}
\begin{align*}
  A_1 = \left(
            \begin{array}{ccc}
              2 &  0 &  0 \\
              0 &  0 & -1 \\
              0 & -1 &  0
            \end{array}
          \right),
  A_2 = \left(
            \begin{array}{ccc}
              0 & 0 & 0 \\
              0 & 1 & 0 \\
              0 & 0 & 0
            \end{array}
          \right),
  A_3 = \left(
            \begin{array}{ccc}
              0 & 0 & 1 \\
              0 & 0 & 0 \\
              1 & 0 & 0
            \end{array}
          \right),
  A_4 = \left(
            \begin{array}{ccc}
              0 & 1 & 0 \\
              1 & 0 & 0 \\
              0 & 0 & 0
            \end{array}
          \right)
\end{align*}
を条件として使用して実験を行った。この実験では双対問題\ref{DualSemidefiniteSystem}の許容解がアルゴリズムの結果として得られる。そのような例としては
\begin{align*}
    0 \cdot \left(
              \begin{array}{ccc}
                2 &  0 &  0 \\
                0 &  0 & -1 \\
                0 & -1 &  0
              \end{array}
            \right)
  + 2 \cdot \left(
              \begin{array}{ccc}
                0 & 0 & 0 \\
                0 & 1 & 0 \\
                0 & 0 & 0
              \end{array}
            \right)
  + 0 \cdot \left(
              \begin{array}{ccc}
                0 & 0 & 1 \\
                0 & 0 & 0 \\
                1 & 0 & 0
              \end{array}
            \right)
  + 0 \cdot \left(
              \begin{array}{ccc}
                0 & 1 & 0 \\
                1 & 0 & 0 \\
                0 & 0 & 0
              \end{array}
            \right)
\end{align*}
となる
\begin{align*}
  \mathbf{u} = \left(
                 \begin{array}{c}
                   0 \\
                   2 \\
                   0 \\
                   0
                 \end{array}
               \right)
\end{align*}
がある。実際、半正定値となるかを確認すると、
\begin{align*}
    0 \cdot \left(
              \begin{array}{ccc}
                2 &  0 &  0 \\
                0 &  0 & -1 \\
                0 & -1 &  0
              \end{array}
            \right)
  + 2 \cdot \left(
              \begin{array}{ccc}
                0 & 0 & 0 \\
                0 & 1 & 0 \\
                0 & 0 & 0
              \end{array}
            \right)
  + 0 \cdot \left(
              \begin{array}{ccc}
                0 & 0 & 1 \\
                0 & 0 & 0 \\
                1 & 0 & 0
              \end{array}
            \right)
  + 0 \cdot \left(
              \begin{array}{ccc}
                0 & 1 & 0 \\
                1 & 0 & 0 \\
                0 & 0 & 0
              \end{array}
            \right)
  = \left(
      \begin{array}{ccc}
        0 & 0 & 0 \\
        0 & 2 & 0 \\
        0 & 0 & 0
      \end{array}
    \right)
\end{align*}
の固有値は
\begin{align*}
  \det \left(\lambda I - \left(\begin{array}{ccc} 0 & 0 & 0 \\ 0 & 2 & 0 \\ 0 & 0 & 0\end{array}\right)\right) = x^2 \left(x - 2\right)
\end{align*}
から$\lambda = 2, 0$でどちらも非負となるので、この行列は半正定値である。

さて、この条件下で作成したプログラムを実際に走らせると図\ref{test1}の実験結果が得られた。なお、横軸はMain Procedureに入力とする$\epsilon$を、縦軸は実行時間を表している。
\begin{figure}
  \centering
  \includegraphics[width=10cm]{test1.png}
  \caption{実験1の実験結果}
  \label{test1}
\end{figure}

理論的には問題(\ref{MainSemidefiniteSystem})の許容解は出ずに、双対問題の許容解(\ref{DualSemidefiniteSystem})が結果として得られるはずであるが、$\epsilon$が大きい間はMain Procedureが先に終了してしまうため、アルゴリズムが解までたどり着くことができない。しかし、ある程度$\epsilon$が小さくなるとMain Procedureも十分な回数分反復することができるようになるため、理論的な結果と一致するようになったと考えられる。実際、図中の縦棒である、$10^{-5}$以降は理論的な結果と一致して双対問題の許容解(\ref{DualSemidefiniteSystem})が得られた。これ以降はどれほど$\epsilon$を小さくしようとも、双対問題の許容解(\ref{DualSemidefiniteSystem})が得られるため、実行時間が一定になっている。

\subsection{実験2}
\begin{align*}
  A_1 = \left(
            \begin{array}{ccc}
              0 &  0 &  0 \\
              0 &  1 & -1 \\
              0 & -1 &  0
            \end{array}
          \right),
  A_2 = \left(
            \begin{array}{ccc}
              0 & 0 & 0 \\
              0 & 1 & 0 \\
              0 & 0 & 0
            \end{array}
          \right),
  A_3 = \left(
            \begin{array}{ccc}
              0 &  0 &  0 \\
              0 & -1 &  0 \\
              0 &  0 & -2
            \end{array}
          \right) \\
\end{align*}
を条件として使用した。この実験では内点実行可能解が存在しないので、双対問題の解\ref{DualSolution}は存在することはなく、またどれほど$\epsilon$を小さくしても主問題(\ref{MainSemidefiniteSystem})の許容解を得ることはできない。

実際に実行して得られた結果は図\ref{test2}の通りである。なお、横軸はMain Procedureに入力とする$\epsilon$を、縦軸は実行時間を表している。
\begin{figure}
  \centering
  \includegraphics[width=10cm]{test2.png}
  \caption{実験2の実験結果}
  \label{test2}
\end{figure}
この結果から、アルゴリズムは内点実行可能解がない場合、$\epsilon$に依存して実行時間が伸びていることがわかる。

\subsection{実験3}
条件として使用する$A_1, A_2, \cdots, A_{30}$をMATLAB標準のrandiを用いて、Algorithm \ref{MakeRandomMatrices}のようにして生成を行った。
\begin{algorithm}
  \caption{$100$次実対称行列群$A_1, A_2, \cdots, A_{30}$の生成}
  \label{MakeRandomMatrices}
  \begin{algorithmic}
    \Input $\text{生成する整数の最小値} \And \text{生成する整数の最大値}$
    \For {$i = 1$ to $30$}
      \State $matrix \leftarrow \text{randi([生成する整数の最小値, 生成する整数の最大値], 100)}$
      \State $A_i \leftarrow matrix + matrix^T$
    \EndFor
  \end{algorithmic}
\end{algorithm}

実験結果は図\ref{randomtest}のようになった。なお、横軸はMain Procedureに入力とする$\epsilon$を、縦軸は実行時間を表している。
\begin{figure}
  \centering
  \includegraphics[width=10cm]{randomtest.png}
  \caption{実験3の実験結果}
  \label{randomtest}
\end{figure}
この実験では$\epsilon$がどのような値であっても問題(\ref{MainSemidefiniteSystem})の許容解$X \succ 0$がみつかる。しかし、前述の実験のように$\epsilon$の値に応じて常に実行時間線形の値となるわけではなく、起伏が激しくなってしまった。これは実行時間が短かすぎたために、コンテキストスイッチなどのアルゴリズムとは本質的には関係ない実行時間による影響が大きくなってしまったのではないかと考えられる。

また、実行した際にえられた許容解の最小固有値を横軸に、縦軸を実行時間とすると、図\ref{eigen}のような結果が得られた。
\begin{figure}
  \centering
  \includegraphics[width=10cm]{eigen.png}
  \caption{最小固有値と実行時間}
  \label{eigen}
\end{figure}
