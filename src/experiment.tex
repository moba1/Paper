アルゴリズムを実装し、各データセットにおけるアルゴリズムの実行時間をそれぞれ示す。実験環境は
\begin{description}
  \item[OS] Mac OS X El Caption (10.11.6)
  \item[CPU] Intel(R) Core(TM) i7-3770 CPU @ 3.40GHz
  \item[Memory] 16GB 1600MHz DDR3
\end{description}
でおこない、Main Procedureに与える小さな実数$\epsilon > 0$は
\begin{algorithm}
  \caption{$epsilon > 0$の与え方}
  \begin{algorithmic}
    \For {$i = 1$ to $10$}
      \For {$j = 9$ to $1$}
        \State $\epsilon = j \times 10^{-i}$
      \EndFor
    \EndFor
  \end{algorithmic}
\end{algorithm}
として、Forループ1反復ごとの$\epsilon$とともにMain Procedureを100回呼び出してその実行時間を計測した。実装はMATLAB 2016bを用いて行った。

\subsection{実験1}
\begin{align*}
  A_1 & = \left(
            \begin{array}{ccc}
              2 &  0 &  0 \\
              0 &  0 & -1 \\
              0 & -1 &  0
            \end{array}
          \right) \\
  A_2 & = \left(
            \begin{array}{ccc}
              0 & 0 & 0 \\
              0 & 1 & 0 \\
              0 & 0 & 0
            \end{array}
          \right) \\
  A_3 & = \left(
            \begin{array}{ccc}
              0 & 0 & 1 \\
              0 & 0 & 0 \\
              1 & 0 & 0
            \end{array}
          \right) \\
  A_4 & = \left(
            \begin{array}{ccc}
              0 & 1 & 0 \\
              0 & 0 & 0 \\
              0 & 0 & 0
            \end{array}
          \right)
\end{align*}
を条件として使用して実験を行った。この実験では双対問題の許容解
\begin{align} \label{test1-results}
  \displaystyle{\sum_{i = 1}^4} u_i A_i \succ 0\text{となる}\mathbf{u}
\end{align}
がアルゴリズムの結果として得られる。

実際に得られた実験結果は図\ref{test1}のようになっている。
\begin{figure}
  \centering
  \includegraphics[width=10cm]{test1.png}
  \label{test1}
  \caption{実験1の実験結果}
\end{figure}

理論的には問題(\ref{MainProblem})の許容解は出ずに、双対問題の許容解(\ref{test1-results})が結果として得られるはずであるが、$\epsilon$が大きい間はMain Procedureが先に終了してしまうため、アルゴリズムが解までたどり着くことができない。しかし、ある程度$\epsilon$が小さくなるとMain Procedureも十分な回数分反復することができるようになるため、理論的な結果と一致するようになったと考えられる。実際、図中の縦棒である、$10^{-5}$以降は理論的な結果と一致して双対問題の許容解(\ref{test1-results})が得られた。これ以降はどれほど$\epsilon$を小さくしようとも、双対問題の許容解(\ref{test1-results})が得られるため、実行時間が一定になっている。

\subsection{実験2}
\begin{align*}
  A_1 & = \left(
            \begin{array}{ccc}
              0 &  0 &  0 \\
              0 &  1 & -1 \\
              0 & -1 &  0
            \end{array}
          \right) \\
  A_2 & = \left(
            \begin{array}{ccc}
              0 & 0 & 0 \\
              0 & 1 & 0 \\
              0 & 0 & 0
            \end{array}
          \right) \\
  A_3 & = \left(
            \begin{array}{ccc}
              0 &  0 &  0 \\
              0 & -1 &  0 \\
              0 &  0 & -2
            \end{array}
          \right) \\
\end{align*}
を条件として使用した。この実験では
\begin{align*}
  \displaystyle{\sum_{i = 1}^4} u_i A_i \succ 0\text{となる}\mathbf{u}
\end{align*}
や、問題\ref{MainProblem}の許容解$X \succ 0$は得ることができない。

実際に得られた結果は図\ref{test2}の通りである。
\begin{figure}
  \centering
  \includegraphics[width=10cm]{test2.png}
  \label{test2-results}
  \caption{実験2の実験結果}
\end{figure}

この実験ではアルゴリズムは双対問題の許容解(\ref{test1-results})や、問題\ref{MainProblem}の許容解$X \succ 0$を得ることはできない。今回の実装として、100回Main Procedureが反復を行った場合、強制的にアルゴリズムを終了するようになっている。そのため、$10^{-6}$以降はこの最大反復回数を超えてしまうため、一定の実行時間しか得られなかった。

\subsection{実験3}
$50$次実対称行列を30個用意した。各要素の値はpythonのメルセンヌツイスタを用いて$[-10,10]$の範囲で生成した整数を用いている。

実験結果は図\ref{test3}のようになった。
\begin{figure}
  \centering
  \includegraphics[width=10cm]{randomtest.png}
  \label{randomtest-results}
  \caption{実験3の実験結果}
\end{figure}

