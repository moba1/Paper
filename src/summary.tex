\subsection{結論}
今回の結果から、ChubanovのアルゴリズムをSDPに拡張したアルゴリズムの実行時間は、許容解がすぐにみつからない場合や、許容解がないような問題の場合にはMain Procedureに入力として渡す$\epsilon$の大きさに依存していると考えられる。対して、許容会があるような場合、すなわち弱実行可能である場合や強実行可能である場合は$\epsilon$は実行時間に寄与せず、行列のサイズや行列の個数などが大きく影響すると考えることができる。したがって、実行可能な場合であるからと行って、あまりに大きく個数の多い行列を解くような場合には注意を払う必要があるだろう。

\subsection{今後の課題}
今回の実験では、データセットとして大きな行列を用いているが、許容解が存在しない場合、すなわち弱実行不可能な場合や強実行不可能な場合について実験を行っていない。したがって、このような場合において、このアルゴリズムの実行時間が$\epsilon$に依存してどのように変化するのか、また行列のサイズや行列の個数なども実行不可能な場合に影響をどの程度及ぼすのかをを確認することができなかった。対称錐上におけるChubanovのアルゴリズムを拡張したアルゴリズムの計算量はすでに求められているので、その計算量から半正定値錐に制限した場合にどの程度の計算量となるのかを慎重に見積もって実験を行っていくべきだろう。

以上から、問題(\ref{MainSemidefiniteSystem})が実行不可能となるような大きなデータセットを用いて実験を行ったとき、どのように$\epsilon$や行列のサイズ、個数に依存するのか、また実行時間はどの程度となるのかを今後の課題としたい。
