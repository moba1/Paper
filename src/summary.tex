\subsection{結論}
今回の結果から、ChubanovのアルゴリズムをSDPに拡張したアルゴリズムの実行時間は、許容解がすぐにみつからない場合や、許容解がないような問題の場合にはMain Procedureに入力として渡す$\epsilon$の大きさに依存していることがわかった。$\epsilon$には結果から線形に依存しているように見えるが、現在ではMain Procedureの計算量が判明していないため、実際に線形かどうかは裏づけに乏しい。

\subsection{今後の課題}
今回の実験では、データセットとして大きな行列を用いるが、許容解が存在しない場合について今回は実験を行っていない。したがって、このような場合において、このアルゴリズムの実行時間が$\epsilon$に依存してどのように変化するのかを確認していない。また、Basic Procedureの計算量を見積もったが、Main Procedureの計算量を見積もっていないため、このアルゴリズムが全体としてどの程度計算が複雑なのかを見積もることができていない。

以上から、
\begin{itemize}
  \item アルゴリズム全体としてどの程度の計算量となるのか
  \item 問題(\ref{SemidefiniteSystem})が実行不可能となるような大きなデータセットを用いて実験を行ったとき、どのように$\epsilon$に依存するのか、また実行時間はどの程度となるのか
\end{itemize}
を今後の課題としたい。
