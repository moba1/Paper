Louren\c{c}oがChubanovのアルゴリズムを対称錐計画問題へ適用できるように拡張している\cite*{SymmetricCone}。対称錐計画問題は特殊ケースとして半正定値計画問題を含んでおり、Louren\c{c}oの提案したアルゴリズムは半正定値計画問題にも使用することができる。そこで本論文ではそのLouren\c{c}oらの提案したアルゴリズムを半正定値計画問題に適用できるよう制限した形で実装を行った。

具体的には、実装したアルゴリズムは半正定値計画問題の許容解を探す
\begin{align} \label{MainSemidefiniteSystem}
  \begin{array}{ll}
    \text{find} & X \\
    \text{s.t.} & A_i \cdot X = 0 \,\, (i = 1, 2, \cdots, m) \\
                & X \succ 0
  \end{array}
\end{align}
という主問題を解いている。この問題の双対問題は
\begin{align} \label{DualSemidefiniteSystem}
  \begin{array}{ll}
    \text{find} & \mathbf{u} \\
    \text{s.t.} & \displaystyle{\sum_{i = 1}^m} u_i A_i \succeq 0 \\
                & \displaystyle{\sum_{i = 1}^m} u_i A_i \not= O
  \end{array}
\end{align}
であり、この双対問題(\ref{DualSemidefiniteSystem})に許容解が存在すれば、主問題(\ref{MainSemidefiniteSystem})に許容解は存在しない。逆に主問題(\ref{MainSemidefiniteSystem})に許容解が存在すれば、双対問題(\ref{DualSemidefiniteSystem})に許容解は存在しない。なぜならば、両方が同時に成立するとすると、系\ref{PsdAndPdInnerProduct}から、
\begin{align*}
  \left(\displaystyle{\sum_{i = 1}^m} u_i A_i\right) \cdot X > 0
\end{align*}
となるが、内積の線形性から
\begin{align*}
  \left(\displaystyle{\sum_{i = 1}^m} u_i A_i\right) \cdot X = \displaystyle{\sum_{i = 1}^m} u_i \left(A_i \cdot X\right)
                                                             = \displaystyle{\sum_{i = 1}^m} u_i \cdot 0
                                                             = 0
\end{align*}
となり、先ほどの$0$より大きくなる結果と矛盾するからである。

\subsection{Basic Procedure}
\subsubsection{アルゴリズム}
Basic Procedureは、
\begin{align*}
  \begin{cases}
      A_1, A_2, \cdots, A_m \in S^n \\
      I \cdot Y_0 = 1 \text{となる} Y_0 \in S_+^n
    \end{cases}
\end{align*}
を入力とし、Louren\c{c}oらの論文における($\mathrm{P}_\mathrm{scaled}^\mathcal{A}$)という問題を半正定値の集合がなす錐へ制限した
\begin{align*}
  \begin{array}{ll}
    \text{find} & X \\
    \text{s.t.} & A_i \cdot X = 0 \,\, \left(i = 1, 2, \cdots, m\right) \\
                & I \cdot Y = 1 \\
                & X \succ 0
  \end{array}
\end{align*}
という問題から解$X$を導こうとする。アルゴリズムが終了すると
\begin{itemize}
  \item $\displaystyle{\|P_\mathcal{A} Y\| \leq \frac{1}{n^{\frac{3}{2}}} \mathrm{tr}\parenthesis(Y)}$となるような$Y \succ 0$
  \item $A_i \cdot X = 0 \,\, \left(i = 1, 2, \cdots, m\right)$となるような$X \succ 0$、すなわち許容解
  \item $\displaystyle{\sum_{i = 1}^m u_i A_i \succ 0}$となるような$\mathbf{u} \in \mathbb{R}^m$
\end{itemize}
のうちの何れかが出力される。$P_\mathcal{A} X$は$X$の$\mathrm{ker}\mathcal{A} := \Set{Y \in S^n | A_i \cdot Y = 0 \,\, \left(i = 1, 2, \cdots, m\right)}$への射影を、$Y_0$はアルゴリズムの開始点をそれぞれ表している。

アルゴリズム全体としてはAlgorithm \ref{BasicProcedure}のようになる。
\begin{algorithm}
  \caption{Basic Procedure}
  \label{BasicProcedure}
  \begin{algorithmic}[1]
    \State $i \leftarrow 0$
    \State $Z \leftarrow P_\mathcal{A} Y_0$
    \While {$Z \not= 0 \And Z \not\succ 0 \And \|Z\| > \displaystyle{\frac{1}{n^\frac{3}{2}} I \cdot Y_i}$}
      \State $C \leftarrow \text{getIdempotent}\left(Z\right)$
      \State $P \leftarrow P_\mathcal{A} C$
      \If {$P \not= O \And P \not\succ 0 \And \|P\| > \displaystyle{\frac{1}{n^{\frac{3}{2}}} I \cdot C}$}
        \State $\alpha \leftarrow \displaystyle{\frac{P \cdot \left(P - Z\right)}{\|Z - P\|^2}}$
        \State $Y_{i + 1} \leftarrow \alpha Y_i + \left(1 - \alpha\right) C$
      \Else
        \State \Return $P$ and $C$
      \EndIf
      \State $i \leftarrow i + 1$
      \State $Z \leftarrow P_\mathcal{A} Y_i$
    \EndWhile
    \State \Return $Z$ and $Y_i$
  \end{algorithmic}
\end{algorithm}

アルゴリズム中のgetIdempotentという関数はAlgorithm \ref{getIdempotent}のようになっている。
\begin{algorithm}
  \caption{getIdempotent$\left(Z\right)$}
  \label{getIdempotent}
  \begin{algorithmic}[1]
    \Input $Z \in S^n \And Z \not= O \And Z \not\succ 0$
    \State $\mathbf{q} \leftarrow Z\text{の固有値の中でもっとも小さなものにおける単位固有ベクトル}$
    \State \Return $\mathbf{q} \mathbf{q}^T$
  \end{algorithmic}
\end{algorithm}

射影$Z = P_\mathcal{A} Y$の計算は次のような$Y$と$Z \in \mathrm{ker} \mathcal{A}$の距離を最小化するような問題
\begin{align*}
  \begin{array}{ll}
    \text{minimize} & \displaystyle{\frac{1}{2}} \|Y - Z\|^2 \\
    \text{s.t.}     & A_i \cdot Z = 0 \,\, \left(i = 1, \cdots, m\right)
  \end{array}
\end{align*}
のKKT条件
\begin{align*}
  \begin{cases}
    Z - Y + \displaystyle{\sum_{i = 1}^m} \lambda_i A_i = O \\
    A_i \cdot Z = 0 \,\, \left(i = 1, \cdots, m\right)
  \end{cases}
\end{align*}
から得られる関係式
\begin{align*}
  A_j \cdot Y = \displaystyle{\sum_{i = 1}^m} \lambda_i A_j \cdot A_i \,\, \left(j = 1, \cdots, m\right)
\end{align*}
を使ったAlgorithm \ref{projection}のように計算する。
\begin{algorithm}
  \caption{$Z = P_\mathcal{A} Y$の計算}
  \label{projection}
  \begin{algorithmic}[1]
    \For {$i = 1$ to $m$}
      \For {$j = 1$ to $m$}
        \State $G_{i j} \leftarrow \mathrm{tr}\left(A_i A_j\right)$
      \EndFor
    \EndFor
    \For {$i = 1$ to $m$}
      \State $b_i \leftarrow A_i \cdot Y$
    \EndFor
    \State $G {\bm \lambda} = \mathbf{b}$を解く。
    \State \Return $Z \leftarrow Y - \displaystyle{\sum_{i = 1}^m \lambda_i A_i}$
  \end{algorithmic}
\end{algorithm}

\subsubsection{計算量}
このアルゴリズムの計算量を示すために、いくつかの補題を示す。
\begin{lemma} \label{NegativeInnerProduct}
  $Y \in S_+^n$の$P_\mathcal{A}$への射影を$Z$とし、$\mathbf{q}$を$Z \in S^n$のもっとも小さい固有値における単位固有ベクトルとする。すると、
  \begin{align*}
    Z \cdot \left(\mathbf{q} \mathbf{q}^T\right) \leq 0
  \end{align*}
  が成立する。
\end{lemma}
\begin{proof}
  $Z$を固有値分解したときの$Z$の固有値$\lambda_1, \lambda_2, \cdots, \lambda_n$が対角成分に並ぶ行列を
  \begin{align*}
    \Lambda = \left(
                \begin{array}{ccc}
                  \lambda_1 &        & \\
                            & \ddots & \\
                            &        & \lambda_n
                \end{array}
              \right)
  \end{align*}
  、その各固有値に対応する単位固有ベクトル$\mathbf{x}_1, \mathbf{x}_2, \cdots, \mathbf{x}_n$を列ベクトルとしたものの行列を
  \begin{align*}
    X = \left(\mathbf{x}_1, \mathbf{x}_2, \cdots, \mathbf{x}_n\right)
  \end{align*}
  とする。このとき、$\lambda_1 \geq \cdots \geq \lambda_n$となるようにする。すると、
  \begin{align*}
    Z \cdot \left(\mathbf{q} \mathbf{q}^T\right) & = \left(X \Lambda X^T\right) \cdot \left(\mathbf{q} \mathbf{q}^T\right) \\
                                                 & = \mathrm{tr}\left(X \Lambda X^T \mathbf{q} \mathbf{q}^T\right) \\
                                                 & = \mathrm{tr}\left(X \Lambda \left(\mathbf{q} \mathbf{q}^T X\right)^T\right) \\
                                                 & = \mathrm{tr}\left(X \Lambda \left(\mathbf{q} \left(0, \cdots, 0, \|\mathbf{q}\|^2\right)\right)^T\right).
  \end{align*}
  $\mathbf{q}$は単位ベクトルなので、
  \begin{align*}
    \|\mathbf{q}\|^2 = 1
  \end{align*}
  なので、
  \begin{align*}
        \mathrm{tr}\left(X \Lambda \left(\mathbf{q} \left(0, \cdots, 0, \|\mathbf{q}\|^2\right)\right)^T\right)
    & = \mathrm{tr}\left(X \Lambda \left(\mathbf{q} \left(0, \cdots, 0, 1\right)\right)^T\right) \\
    & = \mathrm{tr}\left(X \Lambda \left(\begin{array}{c} 0 \\ \vdots \\ 0 \\ 1 \end{array}\right) \mathbf{q}^T\right) \\
    & = \mathrm{tr}\left(X \left(\begin{array}{c} 0 \\ \vdots \\ 0 \\ \lambda_n \|\mathbf{q}\|^2 \end{array}\right) \mathbf{q}^T\right) \\
    & = \mathrm{tr}\left(\lambda_n \mathbf{q} \mathbf{q}^T\right) \\
    & = \mathrm{tr}\left(\lambda_n \|\mathbf{q}\|^2\right) \\
    & = \lambda_n
  \end{align*}
  となり、$\lambda_n < 0$なので、
  \begin{align*}
    Z \cdot \left(\mathbf{q} \mathbf{q}^T\right) \leq 0.
  \end{align*}
\end{proof}
また、$C$は$Z$の最小の固有値に対応する単位固有ベクトル$\mathbf{q}$から生成される$\mathbf{q} \mathbf{q}^T$という行列なので、$q_i \,\, (i = 1, 2, \cdots n)$を$\mathbf{q}$の$i$番目の成分とすると
\begin{align*}
  \|C\|^2 & = C \cdot C \\
          & = \mathrm{tr} \left(\mathbf{q} \mathbf{q}^T \mathbf{q} \mathbf{q}^T\right) \\
          & = \mathrm{tr} \left(\mathbf{q} \mathbf{q}^T\right) \\
          & = \displaystyle{\sum_{i = 1}^m q_i^2} \\
          & = 1
\end{align*}
となる。
\begin{lemma} \label{Shrink}
  $P_\mathcal{A}$を$\mathrm{ker} \mathcal{A} = \Set{Y \in S^n | A_i \cdot Y = 0 \,\, (i = 1, 2, \cdots, m)}$への射影とする。すると、
  \begin{align*}
    \|P_\mathcal{A} X\| \leq \|X\|
  \end{align*}
\end{lemma}
\begin{proof}
  $\|X\|^2$をを考える。すると、
  \begin{align*}
    \|X\|^2 & = X^T X \\
            & = X^T \left(I - P_\mathcal{A} + P_\mathcal{A}\right) X \\
            & = X^T \left(I - P_\mathcal{A}\right) X + X^T P_\mathcal{A} X \\
  \end{align*}
  となる。$\left(I - P_\mathcal{A}\right) = \left(I - P_\mathcal{A}\right)^2$となるので、
  \begin{align*}
    X^T \left(I - P_\mathcal{A}\right) X + X^T P_\mathcal{A} X & = X^T \left(I - P_\mathcal{A}\right) \left(I - P_\mathcal{A}\right) X + X^T P_\mathcal{A} P_\mathcal{A} X\\
                                                               & = \|X^T \left(I - P_\mathcal{A}\right)\|^2 + \|P_\mathcal{A} X\|^2
  \end{align*}
  である。以上から、
  \begin{align*}
    \|P_\mathcal{A} X\| \leq \|X\|
  \end{align*}
  である。
\end{proof}
\begin{lemma} \label{Increment}
  $Z$を、$Y_i \in S_+^n$を$P_\mathcal{A}$で射影した点、$C$を$Z$のもっとも小さな固有値に対応する固有ベクトル$\mathbf{q}$によって生成される行列$\mathbf{q} \mathbf{q}^T$とする。$Y_{i + 1} \in S_+^n$を
  \begin{align*}
    \alpha = \frac{P \cdot (P - Z)}{\|Z - P\|^2}
  \end{align*}
  を用いて、
  \begin{align*}
    Y_{i + 1} = \alpha Y_i + (1 - \alpha) C
  \end{align*}
  とすると、$P_\mathcal{A} Y_{i + 1}$と$P_\mathcal{A} Y_i$の間には
  \begin{align*}
    \frac{1}{\|P_\mathcal{A} Y_{i + 1}\|^2} \geq \frac{1}{\|P_\mathcal{A} Y_i\|^2} + 1
  \end{align*}
  という関係式が成り立つ。
\end{lemma}
\begin{proof}
  $P$を、$P_\mathcal{A}$によって$C$を$\mathrm{ker} \mathcal{A} = \Set{X \in S^n | A_i \cdot Y = 0 \,\,(i = 1, 2, \cdots, m)}$に移した点とする。すると
  \begin{align*}
    Z \cdot P = \mathrm{tr}(Z P) = \mathrm{tr}(Z^T P_\mathcal{A} C) = \mathrm{tr}((P_\mathcal{A} Z)^T C) = (P_\mathcal{A} Z) \cdot C
  \end{align*}
  であり、$P_\mathcal{A}$が射影であることから、$P_\mathcal{A} Z = Z$。なぜならば、$Z$はすでに$\mathcal{A}$上に射影された$Y_i$だからである。したがって、補題\ref{NegativeInnerProduct}から
  \begin{align*}
    (P_\mathcal{A} Z) \cdot C = Z \cdot C = Z \cdot \mathbf{q} \mathbf{q}^T \leq 0
  \end{align*}
  ゆえに、
  \begin{align*}
    \|Z - P\|^2 = \mathrm{tr}((Z - P)(Z - P)) = \mathrm{tr}(Z^2 - 2 P Z + P^2) = \|Z\|^2 - 2 Z \cdot P + \|P\|^2 > 0
  \end{align*}
  となる。また、
  \begin{align*}
    P_\mathcal{A} Y_{i + 1} = \alpha P_\mathcal{A} Y_i + (1 - \alpha) P_\mathcal{A} C = \alpha Z + (1 - \alpha) P
  \end{align*}
  から、
  \begin{align*}
    \|P_\mathcal{A} Y_{i + 1}\|^2 & = (P + \alpha (Z - P)) \cdot (P + \alpha (Z - P)) \\
                                  & = \|P\|^2 + 2 \alpha P \cdot (Z - P) + \alpha^2 \|Z - P\|^2 \\
                                  & = \|P\|^2 - 2 \frac{P \cdot (Z - P)}{\|Z - P\|^2} P \cdot (Z - P) + \frac{(P \cdot (Z - P))^2}{\|Z - P\|^4} \|Z - P\|^2 \\
                                  & = \|P\|^2 - \frac{(P \cdot (Z - P))^2}{\|Z - P\|^2} \\
                                  & = \frac{\|P\|^2 \|Z - P\|^2}{\|Z - P\|^2} - \frac{(P \cdot (Z - P))^2}{\|Z - P\|^2} \\
                                  & = \frac{\|P\|^2 (\|Z\|^2 + \|P\|^2 - 2 P \cdot Z) - (P \cdot Z - \|P\|^2)^2}{\|Z - P\|^2} \\
                                  & = \frac{\|P\|^2 (\|Z\|^2 + \|P\|^2 - 2 P \cdot Z) - ((P \cdot Z)^2 - 2 \|P\|^2 P \cdot Z + \|P\|^4)}{\|Z - P\|^2} \\
                                  & = \frac{\|P\|^2 \|Z\|^2 - (P \cdot Z)^2}{\|Z\|^2 + \|P\|^2 - 2 P \cdot Z} \\
                                  & \leq \frac{\|P\|^2 \|Z\|^2}{\|Z\|^2 + \|P\|^2}
  \end{align*}
  となる。ここで、補題\ref{Shrink}から
  \begin{align*}
    \|P\|^2 = \|P_\mathcal{A} C\|^2 \leq \|C\|^2
  \end{align*}
  であり、
  \begin{align*}
    \|C\|^2 & = C \cdot C \\
            & = \mathrm{tr} \left(C C\right) \\
            & = \mathrm{tr} \left(\left(\mathbf{q} \mathbf{q}^T)\right) \left(\mathbf{q} \mathbf{q}^T)\right)\right) \\
            & = \mathrm{tr} \left(\mathbf{q} \mathbf{q}^T\right) \\
            & = \mathrm{tr} \left(C\right) \\
            & = \mathrm{tr} \left(
                              \begin{array}{ccc}
                                q_1 q_1 & \cdots & q_1 q_n \\
                                \vdots  & \ddots & \vdots \\
                                q_n q_1 & \cdots & q_n q_n
                              \end{array}
                            \right) \\
            & = \displaystyle{\sum_{i = 1}^n} q_i^2 \\
            & = \|\mathbf{q}\|^2 \\
            & = 1
  \end{align*}
  なので、$\|P\|^2 \leq \|C\|^2 = 1$。したがって、
  \begin{align*}
    \frac{1}{\|P_\mathcal{A} Y_{i + 1}\|^2} \geq \frac{1}{\|Z\|^2} + \frac{1}{\|P\|^2} \geq \frac{1}{\|Z\|^2} + 1
  \end{align*}
  が成立する。
\end{proof}

以上の補題からこのアルゴリズムの計算量は$O(n^6)$となる。
\begin{theorem} \label{BasicProcedureOrder}
  Basic Procedureの計算量は$O (n^6)$である。
\end{theorem}
\begin{proof}
  $Y_i \succ 0$を$I \cdot Y_i = 1$となるようにする。すると、$Z \not= 0, P \not= 0$であり、$Z \not\succ 0, P \not\succ 0$であるとき、$\|P_\mathcal{A} Y_i\| \leq \displaystyle{\frac{1}{n^\frac{3}{2}}}$が最も反復回数が多くなる。補題\ref{Increment}から
  \begin{align*}
    \frac{1}{\|P_\mathcal{A} Y_{i + 1}\|^2} \geq \frac{1}{\|Z\|^2} + \frac{1}{\|P\|^2} \geq \frac{1}{\|Z\|^2} + 1
  \end{align*}
  となり、$\displaystyle{\frac{1}{\|P_\mathcal{A} Y_{i + 1}\|^2}}$はすくなくとも1増加するので、最大でもこのアルゴリズムは$n^3$回は反復を行なう。射影$Z$の計算にかかる計算量はGauss Eliminationを用いれば$O (n^3)$で計算することができるので、このアルゴリズムの計算量は$O(n^6)$となる。
\end{proof}

\subsection{Main Procedure}
Main Procedureは
\begin{align*}
  \left\{
    \begin{array}{l}
      A_1, A_2, \cdots, A_m \in S^n \\
      \epsilon > 0
    \end{array}
  \right.
\end{align*}
を入力として受けとり、
\begin{enumerate}[label=(\alph*)]
  \item $\displaystyle{\sum_{i = 1}^m u_i A_i \succeq 0}$となるような$\mathbf{u}$ \label{DualSolution}
  \item $A_i \cdot X = 0 \,\, (i = 1, 2, \cdots, m)$となるような$X \succ 0$。すなわち許容解 \label{FeasibleSolution}
  \item $X \geq \epsilon I, A_i \cdot X = 0 \,\, (i = 1, 2, \cdots, m)$となるような$X$(このような$X$のことを内点実行可能解と呼ぶ)が存在しないこと \label{NoSolution}
\end{enumerate}
のいずれかを出力する。

アルゴリズムはAlgorithm \ref{MainProcedure}のようになる。
\begin{algorithm}
  \caption{Main Procedure}
  \label{MainProcedure}
  \begin{algorithmic}[1]
    \State $\displaystyle{Y_0 \leftarrow \frac{1}{r} I}$
    \State $k \leftarrow 0$
    \State $\epsilon_k \leftarrow 1$
    \For {$i = 1$ to $m$}
      \State $A_i^0 \leftarrow A_i$
    \EndFor
    \State Basic Procedureを$Y_k$と$A_1, A_2, \cdots A_m$を入力として呼び出す \label{CallBasicProcedure}
    \If {Basic Procedureからの返り値が\ref{DualSolution}}
      \State \Return $\mathbf{u}$
    \ElsIf {Basic Procedureからの返り値が\ref{FeasibleSolution}}
      \State \Return $\displaystyle{W_1^{-\frac{1}{2}} W_2^{-\frac{1}{2}} \cdots W_{k - 2}^{-\frac{1}{2}} W_{k - 1}^{-\frac{1}{2}} X W_{k - 1}^{-\frac{1}{2}} W_{k - 2}^{-\frac{1}{2}} \cdots W_2^{-\frac{1}{2}} W_1^{-\frac{1}{2}}}$
    \EndIf
    \State $\displaystyle{W_k \leftarrow \frac{1}{3 I \cdot Y_k} Y_k + (1 - \frac{1}{3 r^\frac{3}{2}}) I}$
    \State $\displaystyle{\epsilon_{k + 1} \leftarrow \frac{\epsilon_k}{\det W_k}}$
    \If {$\epsilon_{k + 1} < \epsilon$}
      \State \Return \ref{NoSolution}
    \EndIf
    \For {$i = 1$ to $m$}
      \State $\displaystyle{A_i \leftarrow W_k^\frac{1}{2} A_i W_k^\frac{1}{2}}$
    \EndFor
    \State $\displaystyle{Y_{k + 1} \leftarrow \frac{1}{I \cdot Y_k} Y_k}$
    \State $k \leftarrow k + 1$
    \State \Goto{CallBasicProcedure}
  \end{algorithmic}
\end{algorithm}
