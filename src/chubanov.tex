Chubanovは線形計画問題
\begin{align*}
  \begin{array}{ll}
    \text{minimize} & \mathbf{c}^T \mathbf{x} \\
    \text{s.t.}     & A \mathbf{x} = \mathbf{0} \\
                    & \mathbf{x} \geq \mathbf{0}
  \end{array}
\end{align*}
の許容解を探し出す強多項式時間アルゴリズムを提案した。このアルゴリズムは2つの手続き
\begin{itemize}
  \item Basic Procedure
  \item Main Procedure
\end{itemize}
から構成されている。

\subsection{Basic Procedure}
このアルゴリズムは$m \times n$行列$A$と$\mathbf{y}^T \mathbf{1} = 1$となるような開始点$\mathbf{y} \in \mathbb{R}^n$を入力として受け取る。この$\mathbf{y}$と$\mathrm{ker}A \Set{\mathbf{x} | A \mathbf{x} = \mathbf{0}}$への射影
\begin{align*}
  P_A = I - A^T \left(A A^T\right)^{-1} A
\end{align*}
を使用して、$\mathbf{x} = P_A \mathbf{y}$を計算し、この$\mathbf{x}$および$\mathbf{y}$を
\begin{align*}
  \alpha = \frac{\mathbf{p}^T \left(\mathbf{p} - \mathbf{x}\right)^T}{\|\mathbf{p} - \mathbf{x}\|^2}
\end{align*}
をもちいて
\begin{align*}
  \mathbf{y}' & = \alpha \mathbf{y} + \left(1 - \alpha\right) \mathbf{e}_i \\
  \mathbf{x}' & = \alpha \mathbf{x} + \left(1 - \alpha\right) \mathbf{p}
\end{align*}
として更新していく。$i$は$x_i \leq 0$となる$\mathbf{x}$のインデックスであり、$\mathbf{p}$は$P_A$の$i$番目の列ベクトル、$\mathbf{e}_i$は$i$番目の要素だけ$1$で、それ以外が$0$の$n$次元横ベクトである。こうすることで、
\begin{itemize}
  \item 許容解$\mathbf{x}^* > 0$
  \item 許容解すべてにおいて$k$番目の要素$x_k^*$が$0$
  \item 許容解すべてにおいて$k$番目の要素$x_k^*$が$\frac{1}{2}$以下になっている
\end{itemize}
のうちどれかを得る事ができる。このアルゴリズムによって得られた解とこの解における$\mathbf{y}$を出力する。このアルゴリズムの計算量は論文中\cite{Chubanov}で$O(n^3)$と示されている。

\subsection{Main Procedure}
このアルゴリズムはは$m \times n$行列$A$を入力とし、Basic Procedureを呼び出して得られる解のうち許容解すべてにおいて$k$番目の要素$x_k^*$が$\frac{1}{2}$以下になっているとき、$A$の$k$列目を$1 / 2$倍し、またBasic Procedureを呼び出す、ということを繰り返す。こうすることで
\begin{itemize}
  \item $A\mathbf{x}^*=\mathbf{b}$となる許容解$\mathbf{x}^* > 0$
  \item 許容解は存在しない
\end{itemize}
のうちどれか1つを導く。このアルゴリズムの計算量は$O\left(n^4 + n^3 L_{\text{min}}\right)$と論文中\cite{Chubanov}ではしめされており、$L_{\text{min}}$はループを一定回数で終了するための上限値である。
