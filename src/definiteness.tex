$n$次実対称行列$A$が半正定値であるとは
\begin{align} \label{PSD}
  \forall \mathbf{x} \in \mathbb{R}^n, \mathbf{x}^T A \mathbf{x} \geq 0
\end{align}
であることをいう。$A$が半正定値であることを示すときは
\begin{align*}
  A \succeq 0
\end{align*}
で表す。また、$A$が正定値であるとは任意の$\mathbf{0}$でない$\mathbf{x} \in \mathbb{R}^n$に対して$A$が
\begin{align} \label{PD}
  \mathbf{x}^T A \mathbf{x} > 0
\end{align}
を満たすことを言う。$A$が正定値であることを示すときは
\begin{align*}
  A \succ 0
\end{align*}
で表す。

$n$次対称行列の集合を$S^n$、$n$次対称行列のなかでも半正定値の集合を$S_+^n$と表記する。半正定値な行列の例としては
\begin{align*}
  \left(
    \begin{array}{cc}
      1 & 0 \\
      0 & 2
    \end{array}
  \right)
\end{align*}
がある。半正定値かどうかを確かめてみると任意の$\mathbf{x} = \left(x_1, x_2\right)^T \in \mathbb{R}^2$に対して
\begin{align*}
  \begin{array}{cc}
         \mathbf{x}^T
           \left(
             \begin{array}{cc}
               1 & 0 \\
               0 & 2
             \end{array}
           \right)
         \mathbf{x}
    =    \left(x_1, 2x_2\right)
           \left(
             \begin{array}{c}
               x_1 \\
               x_2
             \end{array}
           \right)
    =    x_1^2 + 2 x_2^2
    \geq 0
  \end{array}
\end{align*}
となり、半正定値であることがわかる。

また、半正定値の固有値は以下の定理からすべて非負であることがわかる。
\begin{theorem}
  $A$が半正定値ならば$A$の全ての固有値$\lambda_1, \lambda_2, \cdots, \lambda_n$は非負である。
\end{theorem}
\begin{proof}
  $A$の固有値のなかから任意にひとつとってきて$\lambda$とする。この固有値に対応する固有ベクトルを$\mathbf{x}_\lambda$とすると、(\ref{PSD})から
  \begin{align*}
    \mathbf{x}_\lambda^T A \mathbf{x}_\lambda \geq 0
  \end{align*}
  である。$\lambda$は$A$の固有値なので固有値の定義より、
  \begin{align*}
    A \mathbf{x}_\lambda = \lambda \mathbf{x}_\lambda
  \end{align*}
  という関係式が成立するので、
  \begin{align*}
    \mathbf{x}_\lambda^T A \mathbf{x}_\lambda = \mathbf{x}_\lambda^T \lambda \mathbf{x}_\lambda \geq 0.
  \end{align*}
  したがって、
  \begin{align*}
    \mathbf{x}_\lambda^T \lambda \mathbf{x}_\lambda = \lambda \mathbf{x}_\lambda^T \mathbf{x}_\lambda = \lambda \|\mathbf{x}_\lambda\|^2 \geq 0
  \end{align*}
  となる。ここで、$\|\mathbf{x}_\lambda\|^2 \geq 0$なので、$\lambda \geq 0$。
\end{proof}
正定値の場合は(\ref{PD})を使い、同様の手順で簡単に示す事ができる。

逆に、$A$の固有値が全て非負の場合は次の定理から$A$は半正定値であることがわかる。
\begin{theorem}
  $A$の固有値$\lambda_1, \lambda_2, \cdots, \lambda_n$が全て非負ならば$A$は半正定値である。
\end{theorem}
\begin{proof}
  各固有値$\lambda_i \,\, (i = 1, 2, \cdots, n)$に対応する固有ベクトルを$\mathbf{x}_i$とし、その固有ベクトル$\mathbf{x}_i$の共役なベクトル$\mathbf{x}_i^*$とする。$A \mathbf{x}_i$に対して$\mathbf{x}_i^*$を左からかけると
  \begin{align*}
    \mathbf{x}_i^* A \mathbf{x}_i = \mathbf{x}_i^* \lambda_i \mathbf{x}_i = \lambda_i \mathbf{x}_i^* \mathbf{x}_i \geq 0
  \end{align*}
  となり、半正定値の定義を満たしている。したがって、$A$の全ての固有値が非負となる場合、$A$は半正定値となる。
\end{proof}

また、以下のような定理が成立する。
\begin{theorem} \label{PsdMatrix}
  $\mathbf{x} \in \mathbb{R}^n$に対して、$\mathbf{x} \mathbf{x}^T \in S_+^n$である。
\end{theorem}
\begin{proof}
  任意の$\mathbf{x}, \mathbf{y} \in \mathbb{R}^n$に対し、$\mathbf{y}^T \mathbf{x} \mathbf{x}^T \mathbf{y}$を考える。すると
  \begin{align*}
    \mathbf{y}^T \mathbf{x} \mathbf{x}^T \mathbf{y} = \left(\mathbf{x}^T \mathbf{y}\right)^T \mathbf{x}^T \mathbf{y} = \left(\mathbf{x}^T \mathbf{y}\right)^2 \geq 0
  \end{align*}
  となる。よって$\mathbf{x} \mathbf{x}^T \in S_+^n$である。
\end{proof}
が成立する。
