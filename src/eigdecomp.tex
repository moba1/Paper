\subsection{固有値分解}
$n$次実対称行列$A$の固有値$\lambda_1, \lambda_2, \cdots, \lambda_n$は次の定理から実数であることがわかる。
\begin{theorem}
  $n$次実対称行列$A$の固有値$\lambda_1, \lambda_2, \cdots, \lambda_n$は全て実数である。
\end{theorem}
\begin{proof}
  $A$の任意の固有値を$\lambda$と置く。この固有値に対応する固有ベクトルを$\mathbf{x}_\lambda \not= \mathbf{0}$とすると、固有値と固有ベクトルの定義から
  \begin{align} \label{EigenEquation}
    A \mathbf{x}_\lambda = \lambda \mathbf{x}_\lambda.
  \end{align}
  ここで、
  \begin{align*}
    A = \left(
          \begin{array}{ccc}
            a_{1 1} & \cdots & a_{1 n} \\
            \vdots  & \ddots & \vdots \\
            a_{n 1} & \cdots & a_{n n}
          \end{array}
        \right),
    \mathbf{x}_\lambda = \left(
                           \begin{array}{c}
                             x_{\lambda, 1} \\
                             \vdots \\
                             x_{\lambda, n}
                           \end{array}
                         \right)
  \end{align*}
  の共役をそれぞれ
  \begin{align*}
    \overline{A} = \left(
                     \begin{array}{ccc}
                       \overline{a}_{1 1} & \cdots & \overline{a}_{1 n} \\
                       \vdots             & \ddots & \vdots \\
                       \overline{a}_{n 1} & \cdots & \overline{a}_{n n}
                     \end{array}
                   \right),
    \overline{\mathbf{x}_\lambda} = \left(
                                      \begin{array}{c}
                                        \overline{x}_{\lambda, 1} \\
                                        \vdots \\
                                        \overline{x}_{\lambda, n}
                                      \end{array}
                                    \right)
  \end{align*}
  とし、$\lambda$の共役を$\overline{\lambda}$とする。(\ref{EigenEquation})の両辺の共役を取ると
  \begin{align*}
    \overline{A} \overline{\mathbf{x}}_\lambda = \overline{\lambda} \overline{\mathbf{x}}_\lambda
  \end{align*}
  となる。ここで、$A$は$n$次実対称行列であることから、
  \begin{align} \label{ComplexConjugate}
    A \overline{\mathbf{x}}_\lambda = \overline{\lambda} \overline{\mathbf{x}}_\lambda
  \end{align}
  と変形することができる。$A \mathbf{x}_\lambda$に左から$\overline{\mathbf{x}_\lambda}^T$かけたものを考えると、
  \begin{align*}
    \overline{\mathbf{x}}_\lambda^T A \mathbf{x}_\lambda = \overline{\mathbf{x}}_\lambda^T A^T \mathbf{x}_\lambda
                                                         = \left(A \overline{\mathbf{x}}_\lambda\right)^T \mathbf{x}_\lambda
  \end{align*}
  となり、(\ref{ComplexConjugate})から
  \begin{align*}
    \left(A \overline{\mathbf{x}}_\lambda\right)^T \mathbf{x}_\lambda = \left(\overline{\lambda} \overline{\mathbf{x}}_\lambda\right) \mathbf{x}_\lambda
                                                                      = \overline{\lambda} \overline{\mathbf{x}}_\lambda^T \mathbf{x}_\lambda
  \end{align*}
  となる。さらに、(\ref{EigenEquation})から
  \begin{align*}
    \overline{\mathbf{x}_\lambda}^T A \mathbf{x}_\lambda = \overline{\mathbf{x}_\lambda}^T \lambda \mathbf{x}_\lambda
                                                         = \lambda \overline{\mathbf{x}_\lambda}^T \mathbf{x}_\lambda
  \end{align*}
  となることから、
  \begin{align*}
    \overline{\lambda} \overline{\mathbf{x}}_\lambda^T \mathbf{x}_\lambda = \lambda \overline{\mathbf{x}}_\lambda^T \mathbf{x}_\lambda
  \end{align*}
  となり、両辺を$\overline{\mathbf{x}}_\lambda^T \mathbf{x}_\lambda$で割ると、
  \begin{align*}
    \overline{\lambda} = \lambda.
  \end{align*}
  よって固有値の共役が自分自身となるため、$A$の固有値は必ず実数となる。
\end{proof}
また、$n$次実対称行列$A$の固有ベクトルは以下の定理を満たす。
\begin{theorem}
  $n$次実対称行列$A$の固有値に対応する固有ベクトルは互いに直交する。
\end{theorem}
\begin{proof}
  $\lambda_1,\lambda_2$を$A$の固有値とする。すると、
  \begin{align*}
    A \mathbf{x}_1 = \lambda_1 \mathbf{x}_1 \\
    A \mathbf{x}_2 = \lambda_2 \mathbf{x}_2
  \end{align*}
  となる$\lambda_1, \lambda_2$に対応する固有ベクトル$\mathbf{x}_1, \mathbf{x}_2$が存在するが、この各固有ベクトルの共役転置$\mathbf{x}_1^*, \mathbf{x}_2^*$をそれぞれ左からかけると、
  \begin{align*}
    \mathbf{x}_2^* A \mathbf{x}_1 = \lambda_1 \mathbf{x}_2^* \mathbf{x}_1 \\
    \mathbf{x}_1^* A \mathbf{x}_2 = \lambda_2 \mathbf{x}_1^* \mathbf{x}_2
  \end{align*}
  となる。$A$は対称行列なので、
  \begin{align*}
    \mathbf{x}_2^* A \mathbf{x}_1 & = \left(A \mathbf{x}_2\right)^* \mathbf{x}_1 \\
                                  & = \left(\mathbf{x}_1^* A \mathbf{x}_2\right)^* \\
                                  & = \left(\lambda_2 \mathbf{x}_1^* \mathbf{x}_2\right)^* \\
                                  & = \lambda_2 \mathbf{x}_2^* \mathbf{x}_1
  \end{align*}
  となる。したがって、
  \begin{align*}
                    & \lambda_1 \mathbf{x}_2^* \mathbf{x}_1 = \lambda_2 \mathbf{x}_2^* \mathbf{x}_1 \\
    \Leftrightarrow & \left(\lambda_1 - \lambda_2\right) \mathbf{x}_2^* \mathbf{x}_1 = 0.
  \end{align*}
  $\lambda_1 \not= \lambda_2$のとき、$\mathbf{x}_1^* \mathbf{x}_2 = 0$から$\mathbf{x}_1$と$\mathbf{x}_2$は直交する。また、$\lambda_1 = \lambda_2$となるとき、すなわち$A$の固有値$\lambda_i (i = 1, 2, \cdots, n)$が重複度$k$を持つ時、固有値$\lambda_i$についての一般固有空間
  \begin{align*}
    W \left(\lambda_i\right) = \Set{\mathbf{x} \in \mathbb{C}^n | \left(A - \lambda_i I\right)^k\mathbf{x} = \mathbf{0}}
  \end{align*}
  から基底を$k$個持って来れば良い。これは一般に$n$次正方行列の$A$の相違なる固有値を$\lambda_1, \lambda_2, \cdots, \lambda_r \,\, (r \text{は} r \leq n \text{となる非負整数})$とすると、
  \begin{align*}
    \mathbb{C}^n = \displaystyle{\bigoplus_{i = 1}^r} W \left(\lambda_i\right)
  \end{align*}
  と一般固有空間の直和に分解できるためである。
\end{proof}
この事実は$A$の単位固有ベクトルを列ベクトルとした行列$V$が$V^T V = I$という性質を満たす事を述べているので、以下の定理が得られる。
\begin{theorem}
  $A$の固有値を対角成分に並べた行列$\Lambda = \displaystyle{\left(\begin{array}{ccc} \lambda_1 & & \\ & \ddots & \\ & & \lambda_n \end{array}\right)}$とする。$A$は$A$の単位固有ベクトルを列ベクトルとした行列$V$と$\Lambda$を用いて、$A = V \Lambda V^T$に分解される。
\end{theorem}
\begin{proof}
  先ほどの固有ベクトルは直交するという事実を使うと、$V^T V = I$から$V^T = V^{-1}$という関係を導出する事ができる。したがって、
  \begin{align*}
    A V = V \Lambda \Rightarrow A = V \Lambda V^{-1} = V \Lambda V^T.
  \end{align*}
\end{proof}

このように$n$次実対称行列$A$を固有ベクトル$V$と$A$の固有値を対角成分に並べた固有値行列$\Lambda$との積に分解する事を固有値分解という。

\subsection{行列の平方根}
固有値分解を使うと、$n$次実対称行列$A$の平方根$\displaystyle{A^{\frac{1}{2}}}$を定義することができる。$\lambda_1, \lambda_2, \cdots, \lambda_n$を$A$の固有値とすると、
\begin{align*}
  \Lambda^{\frac{1}{2}} = \left(
                            \begin{array}{ccc}
                              \sqrt{\lambda_1} &        & \\
                                               & \ddots & \\
                                               &        & \sqrt{\Lambda_n}
                            \end{array}
                          \right),
\end{align*}
$X$を固有値に対応する固有ベクトル$\mathbf{x}_1, \mathbf{x}_2, \cdots, \mathbf{x}_n$を列に並べたもの、すなわち
\begin{align*}
  X = \left(\mathbf{x}_1, \cdots, \mathbf{x}_n\right)
\end{align*}
とすると、$A^\frac{1}{2}$は
\begin{align*}
  A^\frac{1}{2} = X \Lambda^{\frac{1}{2}} X^T
\end{align*}
と表される。なぜならば2乗すると、
\begin{align*}
  A^{\frac{1}{2}} A^{\frac{1}{2}} & = \left(X \Lambda^{\frac{1}{2}} X^T\right) \left(X \Lambda^{\frac{1}{2}} X^T\right) \\
                                  & = X \Lambda^{\frac{1}{2}} \Lambda^{\frac{1}{2}} X^T \\
                                  & = X \Lambda X^T \\
                                  & = A
\end{align*}
となり、
\begin{align*}
  \Lambda^{\frac{1}{2}} \Lambda^{\frac{1}{2}} & = \left(
                                                    \begin{array}{ccc}
                                                      \sqrt{\lambda_1} &        & \\
                                                                       & \ddots & \\
                                                                       &        & \sqrt{\lambda_n}
                                                    \end{array}
                                                  \right)
                                                  \left(
                                                    \begin{array}{ccc}
                                                      \sqrt{\lambda_1} &        & \\
                                                                       & \ddots & \\
                                                                       &        & \sqrt{\lambda_n}
                                                    \end{array}
                                                  \right) \\
                                              & = \Lambda
\end{align*}
から、元の$A$と一致することからわかる。
