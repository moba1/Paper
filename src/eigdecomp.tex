\subsection{固有値分解}
$n$次実対称行列$A$の固有ベクトルは以下の定理を満たす。
\begin{theorem}
  $n$次実対称行列$A$の相異なる固有値に対応する固有ベクトルは互いに直交する。
\end{theorem}
\begin{proof}
  $\lambda_1,\lambda_2$を$A$の異なる固有値とする。すると、
  \begin{align*}
    A \mathbf{x}_1 = \lambda_1 \mathbf{x}_1 \\
    A \mathbf{x}_2 = \lambda_2 \mathbf{x}_2
  \end{align*}
  となる$\lambda_1, \lambda_2$に対応する固有ベクトル$\mathbf{x}_1, \mathbf{x}_2$が存在するが、この各固有ベクトルの転置$\mathbf{x}_1^T, \mathbf{x}_2^T$をそれぞれ左からかけると、
  \begin{align*}
    \mathbf{x}_2^T A \mathbf{x}_1 = \lambda_1 \mathbf{x}_2^T \mathbf{x}_1 \\
    \mathbf{x}_1^T A \mathbf{x}_2 = \lambda_2 \mathbf{x}_1^T \mathbf{x}_2
  \end{align*}
  となる。$A$は対称行列なので、
  \begin{align*}
    \mathbf{x}_2^T A \mathbf{x}_1 & = \left(A \mathbf{x}_2\right)^T \mathbf{x}_1 \\
                                  & = \left(\mathbf{x}_1^T A \mathbf{x_2}\right) \\
                                  & = \left(\lambda_2 \mathbf{x}_1^T \mathbf{x}_2\right)^T \\
                                  & = \lambda_2 \mathbf{x}_2^T \mathbf{x}_1
  \end{align*}
  となる。したがって、
  \begin{align*}
                    & \lambda_1 \mathbf{x}_2^T \mathbf{x}_1 = \lambda_2 \mathbf{x}_2^T \mathbf{x}_1 \\
    \Leftrightarrow & \left(\lambda_1 - \lambda_2\right) \mathbf{x}_2^T \mathbf{x}_1 = 0
  \end{align*}
  。$\lambda_1 \not= \lambda_2$という仮定から、$\mathbf{x}_2^T \mathbf{x}_1 = 0$。
\end{proof}
この事実は$A$の固有ベクトルを列ベクトルとした行列$V$が$V^T V = I$という性質を満たす事を述べているので、以下の定理が得られる。
\begin{theorem}
  $A$の固有値を対角成分に並べた行列$\Lambda = \displaystyle{\left(\begin{array}{ccc} \lambda_1 & & \\ & \ddots & \\ & & \lambda_n \end{array}\right)}$とする。$A$は$A = V \Lambda V^T$に分解される。
\end{theorem}
\begin{proof}
  先ほどの固有ベクトルは直交するという事実を使うと、$V^T V = I$から$V^T = V^{-1}$という関係を導出する事ができる。したがって、
  \begin{align*}
    A V = V \Lambda \Rightarrow A = V \Lambda V^{-1} = V \Lambda V^T.
  \end{align*}
\end{proof}

このように$n$次実対称行列$A$を固有ベクトル$V$と$A$の固有値を対角成分に並べた固有値行列$\Lambda$との積に分解する事を固有値分解という。

\subsection{行列の平方根}
固有値分解を使うと、$n$次実対称行列$A$の平方根$\displaystyle{A^{\frac{1}{2}}}$を定義することができる。$\lambda_1, \lambda_2, \cdots, \lambda_n$を$A$の固有値とすると、
\begin{align*}
  \Lambda^{\frac{1}{2}} = \left(
                            \begin{array}{ccc}
                              \sqrt{\lambda_1} &        & \\
                                               & \ddots & \\
                                               &        & \sqrt{\Lambda_n}
                            \end{array}
                          \right)
\end{align*}
、$X$を固有値に対応する固有ベクトル$\mathbf{x}_1, \mathbf{x}_2, \cdots, \mathbf{x}_n$を列に並べたもの、すなわち
\begin{align*}
  X = \left(\mathbf{x}_1, \cdots, \mathbf{x}_n\right)
\end{align*}
とすると、$A^\frac{1}{2}$は
\begin{align*}
  A^\frac{1}{2} = X \Lambda^{\frac{1}{2}} X^T
\end{align*}
と表される。なぜならば2乗すると、
\begin{align*}
  A^{\frac{1}{2}} A^{\frac{1}{2}} & = \left(X \Lambda^{\frac{1}{2}} X^T\right) \left(X \Lambda^{\frac{1}{2}} X^T\right) \\
                                  & = X \Lambda^{\frac{1}{2}} \Lambda^{\frac{1}{2}} X^T \\
                                  & = X \Lambda X^T \\
                                  & = A
\end{align*}
となり、
\begin{align*}
  \Lambda^{\frac{1}{2}} \Lambda^{\frac{1}{2}} & = \left(
                                                    \begin{array}{ccc}
                                                      \sqrt{\lambda_1} &        & \\
                                                                       & \ddots & \\
                                                                       &        & \sqrt{\lambda_n}
                                                    \end{array}
                                                  \right)
                                                  \left(
                                                    \begin{array}{ccc}
                                                      \sqrt{\lambda_1} &        & \\
                                                                       & \ddots & \\
                                                                       &        & \sqrt{\lambda_n}
                                                    \end{array}
                                                  \right) \\
                                              & = \Lambda
\end{align*}
から、元の$A$と一致することからわかる。

行列の平方根を用いれば、$A, B \in S_+^n$同士の内積$A \cdot B$は非負になることが次の定理から言える。
\begin{theorem} \label{SemidefiniteInnerProduct}
  $A, B \in S_+^n$とすると
  \begin{align*}
    A \cdot B \geq 0
  \end{align*}
  が成り立つ。
\end{theorem}
\begin{proof}
  $A, B$をそれぞれ
  \begin{align*}
    A = U \Lambda_A U^T \\
    B = V \Lambda_B U^T
  \end{align*}
  と固有値分解する。これらは行列の平方根を用いて、
  \begin{align*}
    A = U \Lambda_A^\frac{1}{2} \Lambda_A^\frac{1}{2} U^T = \left(U \Lambda_A^\frac{1}{2}\right) \left(U \Lambda_A^\frac{1}{2}\right)^T\\
    B = V \Lambda_B^\frac{1}{2} \Lambda_A^\frac{1}{2} V^T = \left(V \Lambda_B^\frac{1}{2}\right) \left(V \Lambda_B^\frac{1}{2}\right)^T
  \end{align*}
  と書き表すことができ、
  \begin{align*}
    A \cdot B & = \mathrm{tr}\left(\left(U \Lambda_A^\frac{1}{2}\right) \left(U \Lambda_A^\frac{1}{2}\right)^T \left(V \Lambda_B^\frac{1}{2}\right) \left(V \Lambda_B^\frac{1}{2}\right)^T\right) \\
              & = \mathrm{tr}\left(\left(U \Lambda_A^\frac{1}{2}\right)^T \left(V \Lambda_B^\frac{1}{2}\right) \left(\left(U \Lambda_A^\frac{1}{2}\right)^T \left(V \Lambda_B^\frac{1}{2}\right)\right)^T\right) \\
              & \geq 0
  \end{align*}
  となるので、$A \cdot B \geq 0$である。
\end{proof}
