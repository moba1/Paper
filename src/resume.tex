\年度{29}
\入学年度{25}
\学籍番号{1311169}
\指導教員{村松 正和}
\氏名{福田 優真}
\題目{ChubanovのアルゴリズムのSDPへの拡張とその実装}
\begin{概要}
  Chubanovhが$A\mathbf{x} = \mathbf{x} = \mathbf{0}, \mathbf{x} \geq \mathbf{0}$という制約を満たす$\mathbf{x}$を探し出すアルゴリズムを提案した。そのアルゴリズムは実際に制約を満たす$\mathbf{x}$を探すBasic ProcedureとBasic Procedureの返り値を元に$A$を適切にスケールして再度Basic Procedureに$\mathbf{x}$を探させるMain Procedureの2つからなる。

  本論文では、このChubanovが提案したアルゴリズムを$\mathrm{tr} \left(A_i X\right) = 0 \left(i = 1, \cdots, m\right)$かつ$\forall \mathbf{y} \in \mathbb{R}^n, \mathbf{y}^T X \mathbf{y} \geq 0$という制約を満たす$n$次実対称行列$X$かこの問題の双対問題の解を探し出すように拡張したアルゴリズムをMATLABで実装し、実装したアルゴリズムで各項目を変化させた時の実行時間を測定する実験を行った。

  実験は解が存在するような場合を1つ、解が存在しない、すなわち双対問題に解が存在するような場合を1つ、正規分布にしたがう乱数によって生成される$100$次実対称行列$30$個を制約に使用するようなものを1つ、さらに弱実行可能な場合と強実行可能な場合それぞれにおいて行列のサイズ、個数、入力として与える$\epsilon$を変化させる実験をそれぞれにおいて1つずつ行った。

  すると、実験結果から、制約によっては解きたい問題の双対問題には解があるような場合でも、$\epsilon$を実験における範囲で小さくしても双対問題の解を見つけることができな場合があることや、弱実行可能あるいは強実行可能な問題に対しては入力の$\epsilon$に依存せず、行列の大きさや行列の個数に大きくアルゴリズムの実行時間が影響を受けていることを確認することができた。
\end{概要}
\writeall

